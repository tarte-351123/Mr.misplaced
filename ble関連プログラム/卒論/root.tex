%
%  愛知工業大学経営情報科学部情報科学科
%    コンピュータシステム専攻用LaTeXテンプレート(2015.12.22)
%
\documentclass[openany]{jbook}

% 使いたい人は使う
\usepackage{personal}

% 図挿入用
\usepackage[dvipdfmx]{graphicx}
\usepackage{latexsym}
\usepackage{amsmath,amssymb}
\usepackage{amsfonts}
\usepackage{ulinej}
\usepackage{url}
\usepackage{mathtools}
\usepackage{here} % 強制的に図を好きな位置に配置するためのパッケージ

\pagestyle{headings}

% カウンタセット
\setcounter{secnumdepth}{3}
\setcounter{tocdepth}{3}

% 定義環境
\newtheorem{definition}{定義}[section]

% 例環境
\newtheorem{example}{例}[section]

% 新しい環境の定義
\newenvironment{indention}[1]{\par
\addtolength{\leftskip}{#1}
\begingroup}{\endgroup\par}

% 関連図書→参考文献
\renewcommand{\bibname}{参考文献}

\topmargin=-14mm
\headsep=15mm
\textwidth=15.7cm
%\baselineskip=22pt
%\renewcommand{\baselinestretch}{1.4}
\textheight=24.5cm  % 33 lines in 1 page
\oddsidemargin=7.5mm
\evensidemargin=7.5mm

% 部分コンパイル用
%\includeonly{title,chap1,chap2,chap3,chap4,chap5,thanks,reference}
\begin{document}

\thispagestyle{myheadings}

\vspace{-1.0cm}

\begin{center}

{\LARGE 愛知工業大学情報科学部情報科学科\\
コンピュータシステム専攻

\vspace{1.0cm}

平成27年度~卒業論文\\

\vspace{2.0cm}

{\Huge 
\baselineskip=15mm
\textbf{オブジェクト指向データに対する\\
グラマーモデルの適用\\}}

\vspace{7.0cm}

2016年2月\\

\vspace{1.0cm}

\begin{tabular}[h]{lll}
  研究者  & K00001 & 愛工太郎\\
         & K00011 & 八草花子\\
         & X00012 & 愛知環状\\
\end{tabular}

\vspace{1.0cm}

指導教員\ \ 情報一郎\ \ 教授}

\end{center}

% Local Variables: 
% mode: latex
% TeX-master: "root"
% End: 


%目次を自動的に作る。
\tableofcontents

% 本文
\chapter{卒業論文の作成と発表の手順}
\thispagestyle{myheadings}

卒業研究は1名(個人)または2〜4名のグループで行う.卒業研究を行う個人
またはグループは以降に示すようにレジュメの執筆,本論文の執筆,口頭発表
を行わなければならない.なお,グループで卒業研究を行う場合は,グループ
で1件,執筆,発表等を行うことになる.

\section{スケジュール}
\label{sec:schedule}

 スケジュールを次に示す.変更等,必要があればco-netを使って連絡される
ので注意する.
\begin{description}
\itemindent=1zw
\itemsep=0mm
\parsep=0mm
\item[1月15日(金)] タイトル提出
\item[1月22日(金)] レジュメ提出
\item[2月5日(金)] 卒業論文締切
\item[2月17日(水)] 卒業論文・制作発表会
\end{description}

\section{レジュメの提出}
\label{sec:abstract}

先に示した締切にしたがって,卒業研究の要旨を記したものを提出する.これ
をレジュメという.レジュメは,指導教員の校閲を受け,提出許可を受けてか
ら,指導教員経由で電子的に提出する.

レジュメは,A4サイズにて提出し,印刷されて,\ulinej{審査にあたる各教員
と4年生及び各研究室に数部に配布される.口頭発表時には,教員や聴講して
いる3年生はこのレジュメを見ながら質問をする.}よく推敲して,研究の要旨
を充分に伝えられるものに仕上げる.指導教員から提出を認められるまで,何
度でも書き直す.レジュメの作成にあたっては \LaTeX やWordなどのソフトウェ
アを使用する.

\section{卒業論文提出}
\label{sec:thesis}

卒業研究の成果は,指導教員の指示に従って,論文の形式にまとめ,それを必
要部数製本する(通常2部).論文は,情報科学科事務室に1部提出する.
提出された論文は,専攻で審査に使われた後,各教員の元で永年にわたって保
存される.卒業論文の提出にあたっては,レジュメの場合と同様に,指導教員
の校閲を受け,また認印を(内表紙に)受ける.

卒業論文のおおよその書式は第2章で説明するが,各研究室のスタイルに従え
ば良い.\LaTeX を利用する場合は,このマニュアルの\LaTeX ソースをテンプレー
トと利用しても良い.

\begin{table}[htbp]
  \centering
  \vspace{10mm}
  \caption{iBeacon管理テーブル}
  \label{tab:paramM}
  \begin{tabular}{lcrrr} \hline\hline
	ビーコン番号&参加型10\%&20\%&30\%&無意識型\\\hline
	1(道幅25m)&24&42&71&238\\
	1(道幅25m)&24&42&71&238\\
	1(道幅25m)&24&42&71&238\\
	1(道幅25m)&24&42&71&238\\\hline
  \end{tabular}
\end{table}

%    ビーコン番号 & 参加型10\% & 20\% & 30\% & 無意識型  \\ \midrule
%    1(道幅25m) & 24 & 42 & 71 & 238 \\ 
  %  2(道幅20m) & 22 & 37 & 76 & 172  \\
    %3(道幅15m) & 16 & 29 & 52 & 71  \\ 
    %4(道幅5m) & 28 & 45 & 78 & 39  \\ \bottomrule

\section{卒業研究口頭発表}
\label{sec:presentation}

卒業研究の仕上げは,卒業研究口頭発表である.卒業研究を1人で行っている
人は個人で,2人以上のグループで行っている人はグループで発表する.
口頭発表は,公開制であり,1〜3年生の聴講も許されている.

口頭発表技術は,話し方,資料のまとめ方,印象的な資料の提示方法等も含む.
口頭発表技術を,ぜひこの機会を利用して習得して欲しい.発表はプレゼンテー
ションソフトを利用して,ノートパソコンと液晶プロジェクターの組み合わせ
で効果的な発表を行う.発表時間は次を基準とする.\\

\begin{tabular}[h]{ll}
  1人で発表を行う場合 & 10分(質疑応答3分を含む)\\
  グループで発表を行う場合 & 15分(質疑応答5分を含む)\\
\end{tabular}

\section{卒業研究の審査・単位認定について}
\label{sec:referee}
審査は論文,レジュメ,口頭発表および日常の研究態度を総合して行われる.

% Local Variables: 
% mode: japanese-LaTeX
% TeX-master: "root"
% End: 


\thispagestyle{myheadings}
\chapter{卒業論文の書式}
\label{sec:format}

\section{論文の書式}
\label{sec:format_thesis}

卒業論文の作成には \LaTeX やWordなどのソフトウェアの使用を原則とする.
通常,論文は2部作成し,1部は情報科学部事務室に提出し,1部は各自が保管
する.事務室に提出する論文は,A4サイズファイルに綴じ,指導教員に確認の
印を押してもらう.綴じ込み用フラットファイルは,1論文につき1冊,学科か
ら各教員へ支給される.ファイルの表紙には論文題目と学籍番号・氏名,指導
教員名を明記する.ファイルを開いたときの第1頁目にも,ファイルの表紙と
同一項目を明記した中表紙をつける.これは,ファイルの外表紙が破損や退色
をしても,内容がわかるようにするためである.

  \begin{tabular}{llcl}
    (1)&提出部数&:&1部(学部事務室に提出)\\
    (2)&用紙    &:&A4サイズ白紙(罫線・枠等なし)\\
    (3)&印字書式& & \\
       &\hspace*{1zw}・印字方向&:&横書き\\
       &\hspace*{1zw}・段組み数&:&1段\\
       &\hspace*{1zw}・文字サイズ&:&10.5ポイント程度\\
       &\hspace*{1zw}・1頁の行数&:&45行程度\\
       &\hspace*{1zw}・頁番号印刷&:&する(ページ外側の上部)\\
       &\hspace*{1zw}・上余白&:&26mm程度\\
       &\hspace*{1zw}・下余白&:&26mm程度\\
       &\hspace*{1zw}・左余白&:&30mm程度\\
       &\hspace*{1zw}・右余白&:&20mm程度\\
    (4)&分量&:&制限なし\\
    (5)&図表&:&・図,表にはそれぞれ通し番号とタイトルをつける\\
          & & &・\underline{表番号およびタイトルは表の上側につける}\\
          & & &・\underline{図番号およびタイトルは図の下側につける}\\
          & & &・グラフには単位および軸の意味を記し,見て理解できるよ
          うにする\\
          & & &・図,表は必ず本文から参照し説明する\\
          & & &\hspace*{1zw}(図,表だけで理解できるのが望ましい)\\
    (6)&見出し&:&・章,節の番号表記,見出しの書き方は別紙に付けたサ
          ンプルに従う\\
          & & &・章が変わるところでは改ページをおこなう\\
          & & &・節が変わるところでは見出しの前に1行あける\\
    (7)&文字種の使い分け&:&・文章中の仮名および漢字は全角とする\\
          & & &・数字,アルファベットは半角とする\\
          & & &・章,節などの見出しは太字などでわかりやすくする\\
          & & &・網掛け,白抜き,倍角等,冗長な強調表現は避ける\\
    (8)&参考文献&:&・参考文献の書き方は付録に示すが,指導教員の
          指示がある場合は,\\
          & & &\hspace*{1zw}それに従うこと\\
    (9)&タイトル等&:&・研究内容を適切に表すタイトルをつける.\\
          & & &\hspace*{1zw}(2行を越えるなど長すぎないように気をつける)\\
          & & &・必要に応じて副題を付けても良いが,この場合も長さに気
                 をつける\\
          & & &・綴じ込み用ファイルにつける外表紙と,中表紙の用意する\\
          & & &\hspace*{1zw}(同じ書式で構わない)\\
          & & &・指導教員から合格を受けたら,中表紙の教員名の右横に印
                 鑑をもらう\\
          & & &・表紙のサンプルを別紙に示す\\
          & & &・目次をつける\\
          & & &・複数人で書いた場合は執筆範囲が担当がわかるようにする\\
  \end{tabular}

\section{論文の構成例}

卒業論文の構成については各指導教員の指示に従うことになるが,参考までに
構成例を次に示す.

\begin{description}
\item[第1章] 「はじめに」または「序論」として,論文概要及び論文の構成
  について説明する.
\item[第2章] 「背景」として論文の背景にあたる内容を書く.
\item[第3章] 「提案手法」として,\textbf{研究の目的}からアプローチまで
  を説明する.章のタイトルが,卒論のタイトルと一致する場合もある.
\item[第4章] 「実験及び考察」提案手法が研究の目的を達成できているかど
  うかを評価確認し,考察する.
\item[第5章] 「まとめと今後の課題」として,達成できたことと,今後の課
  題として取り組むべき内容についてまとめる.
\item[謝辞] お世話になった先生方や先輩達へのお礼を述べる.
\item[参考文献] 卒業論文の執筆に際し,参考にした文献について記述する.
  本文中と\textbf{相互参照}する.
\item[付録] 本文中に入れることが困難であった詳細な定理証明や,アルゴリ
  ズムなどを記述する.
\end{description}
これら,あくまで例であり,各研究室の指導教員の指示に従う.

\section{論文の製本形態と関連資料の整理}
\label{sec:style}

指導教員へ提出する論文の提出形態は指導教員の指示に従う.事務室へ提出す
るものと同等のものを必要とされる場合もあるが,電子的に成果物を提出する
場合もある.また本論文以外のデータやプログラム,作品等についても研究室
に残すものとし,不正等が認められる場合は,厳正な処分が下されることもあ
り得る.
 
\section{レジュメ(概要)の書式}
\label{sec:format_abst}

レジュメの \LaTeX やWordなどのテンプレートは配布されたものを用いて良い.
また,レジュメは電子的にPDFで提出するため,ソフトウェアの種類を限定す
るものではない.

なお,レジュメはテンプレートで示される書式に従うこととするが,伝統的に
は次のような書式が用いられている.ここで示す書式は,一応の目安であり,
1行の文字数や段間の空白については変更が許される.ソフトウェアによって
は,1行の文字数が固定できないものがある.その場合は,ほぼこの書式に準
ずる文字数になるよう,フォントサイズや文字間隔,行間隔を工夫する.

  \begin{tabular}{llcl}
    (1)&提出部数&:&1部(学部事務室に提出)\\
    (2)&用紙    &:&A4サイズ白紙(罫線・枠等なし)\\
    (3)&印字書式& & \\
       &\hspace*{1zw}・印字方向&:&横書き\\
       &\hspace*{1zw}・段組み数&:&2段\\
       &\hspace*{1zw}・表題文字サイズ&:&16ポイント程度\\
       &\hspace*{1zw}・節見出し文字サイズ&:&12ポイント程度\\
       &\hspace*{1zw}・本文文字サイズ&:&10ポイント程度\\
       &\hspace*{1zw}・1頁の行数&:&45行程度\\
       &\hspace*{1zw}・頁番号印刷&:&しない\\
       &\hspace*{1zw}・上余白&:&15mm程度\\
       &\hspace*{1zw}・下余白&:&20mm程度\\
       &\hspace*{1zw}・左余白&:&15mm程度\\
       &\hspace*{1zw}・右余白&:&15mm程度\\
    (4)&分量&:&2頁\\
    (5)&図表&:&・論文の書式に準ずるものとする\\
    (6)&見出し&:&・論文の書式に準ずるが,章が変わるところでは,大見出し
        の前に\\
          & & &\hspace*{1zw}1行あけるものとする\\
    (7)&文字種の使い分け&:&・論文の書式に準ずるものとする\\
    (8)&参考文献&:&・論文の書式に準ずるものとする\\
    (9)&タイトル等&:&・研究内容を適切に表すタイトルをつける.\\
          & & &\hspace*{1zw}(2行を越えるなど長すぎないように気をつける)\\
          & & &・必要に応じて副題を付けても良いが,この場合も長さに気
          をつける\\
          & & &・タイトルに続いて学籍番号,名前をつける\\
          & & &・最後に指導教員名をつける\\
          & & &・指導教員から合格を受けたら,教員名の右横に印鑑をもらう\\
  \end{tabular}

\clearpage

% Local Variables: 
% mode: japanese-LaTeX
% TeX-master: "root"
% End: 


\chapter{オブジェクト指向データに対するグラマーモデルの適用}
\thispagestyle{myheadings}

\section{オブジェクト指向データベース}

\subsection{オブジェクト指向データベースシステムの背景}

\subsubsection{背景}

\subsubsection{従来のデータベースシステム}

\subsubsection{意味データモデル技術}

\subsubsection{オブジェクト指向プログラミング言語}

\subsection{オブジェクト指向データベースシステムとは何か}

\subsection{まとめ}

\section{グラマーモデルへの適用}

\subsection{グラマーモデルへの適用方法の提案}

\subsection{データ構造}

\subsection{検索アルゴリズム}

\subsection{登録アルゴリズム}

\subsection{削除アルゴリズム}

% Local Variables: 
% mode: japanese-LaTeX
% TeX-master: "root"
% End: 


\chapter{グラマーモデルデータベースシステムでの適用例}
\thispagestyle{myheadings}

\section{例題}

\section{実行結果}

\subsection{データの登録}

\subsection{データの検索}

\subsection{導出データを条件とした検索}

\subsection{データの削除}

% Local Variables: 
% mode: japanese-LaTeX
% TeX-master: "root"
% End: 


\chapter{おわりに}
\thispagestyle{myheadings}

\section{まとめ}

\section{今後の課題}

% Local Variables: 
% mode: japanese-LaTeX
% TeX-master: "root"
% End: 


\chapter*{謝辞の例}
\addcontentsline{toc}{chapter}{\protect\numberline {} 謝辞の例}

本研究を進めるにあたり,多くの御指導,御鞭撻を賜わりました
愛知工大教授に深く感謝致します.

また,御討論、御助言していただきました,
○×大学工学部電子情報工学科の山谷川介教授,および山谷研究室のみなさん
に深く感謝致します.

最後に,日頃から熱心に討論,助言してくださいました
愛知研究室のみなさんに深く感謝致します.

% Local Variables: 
% mode: latex
% TeX-master: "root"
% End: 


\thispagestyle{myheadings}
\addcontentsline{toc}{chapter}{\protect\numberline {} 参考文献の書き方例}
    
\begin{thebibliography}{参考文献}

\bibitem{Colby}
Latha, S. Colby and Dirk VanGucht,
``A Grammar Model for Database'',
{\it TECHNICAL REPORT,} NO.282,
June 1989.

\bibitem{Gonnet}
Gaston, H. Gonnet and Frank Wm. Tompa,
``Mind Your Grammar: a New Approach to Modelling Text'',
{\it Proceedings of the 13th VLDB Conference,}Brighton,
pp. 339--346, 1987.

\bibitem{Dzenan}
Dzenan RIDJANOVIC and Micheal L. BRODIE,
``DEFINING DATABASE DYNAMICS WITH ATTRIBUTE GRAMMARS'',
{\it INFORMATION PROCESSING LETTERS,}vol. 14, No. 3,
pp. 132--138, May 1982.


%\bibitem{ifo}
%Abiteboul, S. and Hull, R.
%``IFO: A formal semantic database model''
%{\sl ACM Transactions on Database Systems  12,} 4,
%December 1987,
%pp.525-565.
%
%\bibitem{er}
%Chen, P. P. 
%``The entitiy-relationship model-toward a unified view of data''
%{\sl ACM Transactions on Database Systems 1,} 1,
%March 1976,
%pp.9-36.
%
%\bibitem{sdm}
%Hammer, M. and McLeod, D.
%``Database discription with SDM: A semantic database model''
%{\sl ACM Transactions on Database Systems 6,} 3,
%September 1981,
%pp.351-386.
%
%\bibitem{fdm}
%Shipman, D. W.
%``The functioal data model and the language DAPLEX''
%{\sl ACM Transactions on Database Systems 6,} 1,
%March 1981,
%pp.140-173.
%
%\bibitem{o2}
%Bancilhon, F., et al.
%``The design and implementation of O$_2$, an object-oriented database system''
%{\sl Processings of 2nd International Workshop on Object-Oriented Database Systems}
%1988,
%pp.1-22.
%
%\bibitem{gem}
%Copeland, G., and Maier, D.
%``Making Smalltalk a database system''
%{\sl Processings of the Annual SIGMOD Conference,}
%June 1984.
%
%\bibitem{iris}
%Fishman, D. H., et al. 
%``Iris: An object oriented database management system''
%{\sl ACM Transactions on Database Systems 5,} 1,
%January 1987,
%pp.48-69.
%
%\bibitem{orion}
%Kim, W.,Manerjee, J., Chou, H. T., Garza, J. F., and Woelk, D.
%``Composite object support in an object-oriented database system''
%{\sl Proceedings of OOPSLA,}
%1987,
%pp.118-125.

\bibitem{Hull}
Hull, R. and Yap, C. K.
``The Format model: A theory of database organization'',
{\it JACM 31,}3,
pp. 518--537, 1984.

\bibitem{jap1}
増永良文,
``次世代データベースシステムとしてのオブジェクト指向データベースシステム'',
情報処理,Vol. 32, No. 5,
pp. 490--499, May 1991.

% \bibitem{jap2}
% 田中克己,
% ``オブジェクト指向データベースの基礎概念''
% 情報処理,Vol. 32, No. 5,
% May 1991,
% pp. 500-513.

\bibitem{omt}
J.ランボー M.プラハ W.プレメラニ F.エディ W.ローレンセン,
``OBJECT-ORIENTED MODELING AND DESIGN'',
トッパン, 1992.

% \bibitem{model}
% 米澤明憲,
% ``モデル化と表現''
% 岩波書店(1992).

% \bibitem{prolog}
% 中島秀之,
% ``Prolog''
% 産業図書(1983).

\end{thebibliography}
% Local Variables: 
% mode: japanese-LaTeX
% TeX-master: "root"
% End: 


% これ以降,付録となる
\appendix

\end{document}
